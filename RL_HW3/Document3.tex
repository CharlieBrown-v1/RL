\documentclass[a4paper,12pt]{article}
\usepackage{times}  % DO NOT CHANGE THIS
\usepackage{helvet} % DO NOT CHANGE THIS
\usepackage{courier}  % DO NOT CHANGE THIS
\usepackage[hyphens]{url}  % DO NOT CHANGE THIS
\usepackage{graphicx} % DO NOT CHANGE THIS
\urlstyle{rm} % DO NOT CHANGE THIS
\def\UrlFont{\rm}  % DO NOT CHANGE THIS
\usepackage{natbib}  % DO NOT CHANGE THIS AND DO NOT ADD ANY OPTIONS TO IT
\usepackage{caption} % DO NOT CHANGE THIS AND DO NOT ADD ANY OPTIONS TO IT
\frenchspacing  % DO NOT CHANGE THIS
\setlength{\pdfpagewidth}{8.5in}  % DO NOT CHANGE THIS
\setlength{\pdfpageheight}{11in}  % DO NOT CHANGE THIS
\usepackage{algorithm} %format of the algorithm 
\usepackage{algorithmic} %format of the algorithm 
\usepackage{multirow} %multirow for format of table 
\usepackage{amsmath} 
\usepackage{xcolor}
\usepackage{amssymb}
\usepackage{enumerate}
\usepackage{amsmath}
\usepackage{xeCJK}
\usepackage{courier}
\DeclareMathOperator*{\argmax}{argmax} % thin space, limits underneath in displays

\begin{document}

\title{强化学习:作业三}

\author{杨思航 191180166}

\date{\today}

\maketitle

\section{作业内容}
在Atari环境下实现Deep Q-learning Network算法。

\section{实验环境}
\begin{itemize}
    \item NAME = Ubuntu
    \item VERSION = 20.04.2 LTS(Focal Fossa)
\end{itemize}

\section{实现过程}
\begin{enumerate}
    \item 超参修改
    \begin{enumerate}[(1)]
        \item 修改原则: 最大化 DQN 性能
        \item learning\_rate: $1e-6\to 1e-4$
        \item learning\_interval: $4\to 1$
        \item 优化器:
        \begin{itemize}
            \item RMSprop $\to$ Adam
            \item eps: $1e-5\to 1e-3$
            \item 移除参数 weight\_decay, momentum, centered
        \end{itemize}
    \end{enumerate}
    \newpage
    \item DQN (后续代码均仅列出相较于 DQN 而言修改过的部分)
    \begin{itemize}
        \item Q
        \begin{figure*}[h!]
            \centering
            \includegraphics[scale=0.5]{pics/dqn-Q.png}
        \end{figure*}
        \item target-Q
        \begin{figure*}[h!]
            \centering
            \includegraphics[scale=0.3]{pics/dqn-tQ.png}
        \end{figure*}
        \item loss (loss\_func = MSELoss)
        \begin{figure*}[h!]
            \centering
            \includegraphics[scale=0.5]{pics/dqn-loss.png}
        \end{figure*}
    \end{itemize}
    \item Double DQN
    \begin{itemize}
        \item target-Q
        \begin{figure*}[h!]
            \centering
            \includegraphics[scale=0.3]{pics/ddqn-tQ.png}
        \end{figure*}
    \end{itemize}
    \newpage
    \item Dueling DQN
    \begin{itemize}
        \item value
        \begin{figure*}[h!]
            \centering
            \includegraphics[scale=0.35]{pics/dueldqn-val.png}
        \end{figure*}
        \item advantage
        \begin{figure*}[h!]
            \centering
            \includegraphics[scale=0.35]{pics/dueldan-advan.png}
        \end{figure*}
        \item forward
        \begin{itemize}
            \item From Lecture 8 Page 14
            \begin{figure*}[h!]
                \centering
                \includegraphics[scale=0.35]{pics/dueldqn-version.png}
            \end{figure*}
            \item 代码
            \begin{figure*}[h!]
                \centering
                \includegraphics[scale=0.4]{pics/dueldqn-for.png}
            \end{figure*}
        \end{itemize}
    \end{itemize}
\end{enumerate}

\section{实验效果}
\begin{enumerate}
    \item DQN
    \begin{itemize}
        \item train 
        \begin{enumerate}
            \item 运行结果
            \begin{figure*}[h!]
                \centering
                \includegraphics[width=12cm, height=0.5cm]{pics/dqn-ada.png}
            \end{figure*}
            \item tensorboard
            \begin{figure*}[h!]
                \centering
                \includegraphics[scale=0.28]{pics/dqn-t0.png}
            \end{figure*}
            \begin{figure*}[h!]
                \centering
                \includegraphics[scale=0.28]{pics/dqn-t1.png}
            \end{figure*}
        \end{enumerate}
        \newpage
        \item test
        \begin{figure*}[h!]
            \centering
            \includegraphics[width=8cm, height=0.7cm]{pics/dqn-test.png}
        \end{figure*}
        \item gif 见 dqn.gif
    \end{itemize}
    \item Double DQN
    \begin{itemize}
        \item train 
        \begin{enumerate}
            \item 运行结果
            \begin{figure*}[h!]
                \centering
                \includegraphics[width=12cm, height=0.5cm]{pics/ddqn-ada.png}
            \end{figure*}
            \item tensorboard
            \begin{figure*}[h!]
                \centering
                \includegraphics[scale=0.28]{pics/ddqn-t0.png}
            \end{figure*}
            \newpage
            \begin{figure*}[h!]
                \centering
                \includegraphics[scale=0.28]{pics/ddqn-t1.png}
            \end{figure*}
        \end{enumerate}
        \item test
        \begin{figure*}[h!]
            \centering
            \includegraphics[width=8cm, height=0.7cm]{pics/ddqn-test.png}
        \end{figure*}
        \item gif 见 ddqn.gif
    \end{itemize}
    \item Dueling DQN
    \begin{itemize}
        \item train 
        \begin{enumerate}
            \item 运行结果
            \begin{figure*}[h!]
                \centering
                \includegraphics[width=12cm, height=0.5cm]{pics/dueldqn-ada.png}
            \end{figure*}
            \newpage
            \item tensorboard
            \begin{figure*}[h!]
                \centering
                \includegraphics[scale=0.28]{pics/dueldqn-t0.png}
            \end{figure*}
            \begin{figure*}[h!]
                \centering
                \includegraphics[scale=0.28]{pics/dueldqn-t1.png}
            \end{figure*}
        \end{enumerate}
        \item test
        \begin{figure*}[h!]
            \centering
            \includegraphics[width=12cm, height=1cm]{pics/dueldqn-res.png}
        \end{figure*}
        \item gif 见 dueldqn.gif
    \end{itemize}
\end{enumerate}

\newpage
\section{复现实验}
\begin{itemize}
    \item 为方便助教复现, 我已将代码整合成3份, 助教可于各文件夹内通过 tensorboard 或 $--$test 进行复现 
    \item DQN
    \begin{itemize}
        \item tensorboard
        \begin{figure*}[h!]
            \centering
            \includegraphics[width=12cm, height=1cm]{pics/dqn-t-re.png}
        \end{figure*}
        \item test
        \begin{figure*}[h!]
            \centering
            \includegraphics[width=15cm, height=1cm]{pics/dqn-test-re.png}
        \end{figure*}
    \end{itemize}
    \item Doublu DQN
    \begin{itemize}
        \item tensorboard
        \begin{figure*}[h!]
            \centering
            \includegraphics[width=12cm, height=1cm]{pics/ddqn-t-re.png}
        \end{figure*}
        \item test
        \begin{figure*}[h!]
            \centering
            \includegraphics[width=15cm, height=1cm]{pics/ddqn-test-re.png}
        \end{figure*}
    \end{itemize}
    \item Dueling DQN
    \begin{itemize}
        \item tensorboard
        \begin{figure*}[h!]
            \centering
            \includegraphics[width=12cm, height=1cm]{pics/dueldqn-t-re.png}
        \end{figure*}
        \item test
        \begin{figure*}[h!]
            \centering
            \includegraphics[width=15cm, height=1cm]{pics/dueldqn-test-re.png}
        \end{figure*}
    \end{itemize}
\end{itemize}

\section{小结}
\begin{itemize}
    \item 本次作业帮助我掌握了 tensorflow 与 pytorch 的基本使用, 加深了对深度强化学习的理解
    \item weight\_decay 设置过大会造成模型训练时过于“守旧”, 无法习得新能力, 导致训练效果不良
    \item 从实验结果可以看出: 相较于 DQN, Double DQN 虽然加速了训练, 但是训练曲线震荡更剧烈, 具体测试表现也稍弱于 DQN; 而 Dueling DQN 则在稳定性上有较大突破, 但是在我设置(专为 DQN 调整)的超参下训练稍慢于 DQN
\end{itemize}

\end{document}